% \iffalse
%<*driver>
%<*gobble>
\ifx\jobname\relax\let\documentclass\undefined\fi
\ifx\documentclass\undefined\else \csname fi \endcsname
%</gobble>
\documentclass{ltxdoc}
	\EnableCrossrefs
\RecordChanges
\typeout{Expect some Under- and overfull boxes}
\begin{document}
\DocInput{kuthesis.dtx}
\end{document}
%<*gobble>
\fi
%</gobble>
%</driver>
%
%<*main>
% \fi
%
% \def\fileversion{v0.5} \def\filedate{2015/04/15}
%
% \title{{kuthesis.dtx}\thanks{This file has version \fileversion\ last revised \filedate}}
% \author{Arthur Shipkowski and Alison Chan}
% \maketitle
%
%	\begin{abstract}
%	Kuthesis.dtx is a style file for the Kettering University Thesis Format for \LaTeX.
%	It is designed to cover most needs that Kettering thesis students have.
%	\end{abstract}
%
% \section*{Preface}
%
% \paragraph{Usage of other styles.}
%	You will find traces of other style files sucked in wholesale by
% this file.  This is not to offend the original owners of those files; I
% figure that a student adventurous enough to try this file out ought to
% have all the help they can get.  The most flagrant use of this is in the endnote
% section.  Due to the multitude of available files for to do it, I have not
% added APA-style citing.
%
% \paragraph{Special note.}
%  This style does not attempt to provide
% APA-style citing. There are a number of packages out there,
% of which the most comprehensive is apacite (which has both
% a \LaTeX\ style and a \BibTeX\ style).  However, you'll have to
% |\renewcommand| the keyword |\bibname|, because they define it to
% have lowercase and use addcontentsline to get around any
% provisions I may make.
%
% \paragraph{Arthur's thanks.} There are three people I wish to thank for help on creating this file.  
% First, Professor Mark Wicks of Kettering University, who answered a good number of my
% headscratchers.  Second, Professor James McDonald of Kettering University,
% who encouraged me to use \LaTeX\ to create my thesis.  Finally, my fiance\'{e}, Stephanie
% Rybicki, who was kind enough to give me the kick I needed to get this 
% file documented.  It is still somewhat of a crawling horror, though.
% \paragraph{Alison's thanks.} I'd like to thank Mark Wicks for digging up Arthur's work over a 
% decade after the fact, and Arthur for not thinking it creepy that I contacted him out of the 
% blue regarding this project, and for permitting me to host it on Github and update it for the 
% current thesis specifications.
%
% \section{Getting Started}
%
% \subsection{Requirements}
%
% This style file requires the \texttt{setspace}, \texttt{times},
% \texttt{helvet}, \texttt{ifthen}, and \texttt{calc} packages.  With the
% possible exception of \texttt{setspace}, all of these should be in your
% \TeX\ distribution.
%
% \subsection{Installation}
%
% \begin{enumerate}
% \item	Make sure the files are in your current working directory and run \texttt{kuthesis.ins} through \TeX.
% \item Move the generated files to someplace you'd like them to be.  I'm sure there's an official spot.
% \item If your \TeX\ implementation uses a filename database, update it.
% \end{enumerate} 
%
% \subsection{Usage}
%
% The use of this file is actually fairly simple, as it acts almost
% exactly like the report style.  Just add the following to your
% file:\\
% \\
% |\documentclass[12pt]{kuthesis}      % Specifies the document class |
% \\
% \\
% Of course, in order to completely make a file fit the Kettering Thesis
% manual, there is more to do than that.  First off, in addition
% to |\title{}| and |\author{}|, you are to define other things.
% To start with, you must define your department with 
% |\department{Computer Engineering}|.  Then, define your employer with
% |\employer{MAXIMUM WIDGETS}|.  Finally, define your graduation
% year with |\gdate{2015}|.  This way, when you use
% |\maketitle|, not only will your title page be generated but your
% disclaimer page will be generated as well.
%
% Next, enclose your preface in the preface environment with |\begin{preface}|
% and |\end{preface}|.  
%
% In order to produce the table of contents, list of illustrations, and list
% of tables, you'll need to do the following.
%
%
% \begin{verbatim}
% \singlespacing
% \tableofcontents\ 
% \listofillustrations
% \end{verbatim}
%
% Finally, when it is time to begin the body of your thesis,
% use |\begincontent|.  Enter chapters as you would for a report.
% Note that footnotes are not permitted; use |\endnote| with a |\theendnotes|
% at the end of each chapter.
%
% Creating a glossary will require you to |\setcounter{secnumdepth}{-1}|
% before the glossary and |\setcounter{secnumdepth}{0}| afterwards.  Declare
% appendicies with |\appendix| before chapter names, but note that a 
% single appendix will have to be done by hand.
%
% \subsection{Using \texttt{apacite} with \texttt{kuthesis.cls}}
%
% Just use the following code, and do citations as normal.  Naturally,
% don't forget to use the apacite package in your preamble.
%
% \begin{verbatim}
% \begin{singlespace}
% \bibliographystyle{apacite}
% \bibliography{thesis}
% \end{singlespace}
% \end{verbatim}
%
% \section{Source Code Examination and Discussion}
%
% This source code is based heavily on the report style source code.  Therefore,
% I will focus more on commenting on changes and interesting properties.
%
%    \begin{macrocode}
\NeedsTeXFormat{LaTeX2e}[1995/12/01]
\ProvidesClass{kuthesis}[2015/04/15 Kettering University Thesis class definition]
%    \end{macrocode}
%
% We must double space almost everything and put it in 12 pt Times New Roman.
% In certain cases, Helvetica may be needed.  Since we will still occasionally
% need to single space things, we simply use the \texttt{setspace} package.
%
%    \begin{macrocode}
\RequirePackage{setspace}
\RequirePackage{times}
\RequirePackage{helvet}
\RequirePackage{ifthen}
\RequirePackage{calc}
\RequirePackage{etoolbox}
%    \end{macrocode}
% There are spots in this file where options remain that aren't useful.
% Paper sizes are one of these things.  One of these days perhaps someone
% will rip it out, but I'm not too willing to risk yet another style change.
%    \begin{macrocode}
\newcommand\@ptsize{}
\newif\if@restonecol
\newif\if@titlepage
\@titlepagetrue
\newif\if@openright
\if@compatibility\else
\DeclareOption{a4paper}
   {\setlength\paperheight {297mm}%
    \setlength\paperwidth  {210mm}}
\DeclareOption{a5paper}
   {\setlength\paperheight {210mm}%
    \setlength\paperwidth  {148mm}}
\DeclareOption{b5paper}
   {\setlength\paperheight {250mm}%
    \setlength\paperwidth  {176mm}}
\DeclareOption{letterpaper}
   {\setlength\paperheight {11in}%
    \setlength\paperwidth  {8.5in}}
\DeclareOption{legalpaper}
   {\setlength\paperheight {14in}%
    \setlength\paperwidth  {8.5in}}
\DeclareOption{executivepaper}
   {\setlength\paperheight {10.5in}%
    \setlength\paperwidth  {7.25in}}
\DeclareOption{landscape}
   {\setlength\@tempdima   {\paperheight}%
    \setlength\paperheight {\paperwidth}%
    \setlength\paperwidth  {\@tempdima}}
\fi
\if@compatibility
  \renewcommand\@ptsize{0}
\else
\DeclareOption{10pt}{\renewcommand\@ptsize{0}}
\fi
\DeclareOption{11pt}{\renewcommand\@ptsize{1}}
\DeclareOption{12pt}{\renewcommand\@ptsize{2}}
\if@compatibility\else
\DeclareOption{oneside}{\@twosidefalse \@mparswitchfalse}
\fi
\DeclareOption{twoside}{\@twosidetrue  \@mparswitchtrue}
\DeclareOption{draft}{\setlength\overfullrule{5pt}}
\if@compatibility\else
\DeclareOption{final}{\setlength\overfullrule{0pt}}
\fi
\DeclareOption{titlepage}{\@titlepagetrue}
\if@compatibility\else
\DeclareOption{notitlepage}{\@titlepagefalse}
\fi
\if@compatibility
\else
\DeclareOption{openright}{\@openrighttrue}
\DeclareOption{openany}{\@openrightfalse}
\fi
\if@compatibility\else
\DeclareOption{onecolumn}{\@twocolumnfalse}
\fi
\DeclareOption{twocolumn}{\@twocolumntrue}
\DeclareOption{leqno}{\input{leqno.clo}}
\DeclareOption{fleqn}{\input{fleqn.clo}}
\DeclareOption{openbib}{%
  \AtEndOfPackage{%
   \renewcommand\@openbib@code{%
      \advance\leftmargin\bibindent
      \itemindent -\bibindent
      \listparindent \itemindent
      \parsep \z@
      }%
   \renewcommand\newblock{\par}}%
}
%    \end{macrocode}
% The following options are what you get if you do not specify any
% options in the class definition. It is designed to produce a thesis
% the way the university wants to see it.
%    \begin{macrocode}
\ExecuteOptions{letterpaper,12pt,oneside,onecolumn,final,openany}
\ProcessOptions
\input{size1\@ptsize.clo}
%    \end{macrocode}
% To produce output with the desired line spacing, the argument of
% |\spacing| should be multiplied by $5/6 = 0.8333$, so that $1\frac{1}{2}$ spacing
% corresponds to |\spacing{1.5}| and double spaced is |\spacing{1.66}|.
%    \begin{macrocode}
\def\normalspacing{1.25} % default line spacing
\setlength\lineskip{1\p@}
\setlength\normallineskip{1\p@}
\renewcommand\baselinestretch{}
\setlength\parskip{0\p@ \@plus \p@}
\@lowpenalty   51
\@medpenalty  151
\@highpenalty 301
\setcounter{topnumber}{2}
\renewcommand\topfraction{.7}
\setcounter{bottomnumber}{1}
\renewcommand\bottomfraction{.3}
\setcounter{totalnumber}{3}
\renewcommand\textfraction{.2}
\renewcommand\floatpagefraction{.5}
\setcounter{dbltopnumber}{2}
\renewcommand\dbltopfraction{.7}
\renewcommand\dblfloatpagefraction{.5}
\if@twoside
  \def\ps@headings{%
      \let\@oddfoot\@empty\let\@evenfoot\@empty
      \def\@evenhead{\thepage\hfil\slshape\leftmark}%
      \def\@oddhead{{\slshape\rightmark}\hfil\thepage}%
      \let\@mkboth\markboth
    \def\chaptermark##1{%
      \markboth {\MakeUppercase{%
        \ifnum \c@secnumdepth >\m@ne
            \@chapapp\ \thechapter. \ %
        \fi
        ##1}}{}}%
    \def\sectionmark##1{%
      \markright {\MakeUppercase{%
        \ifnum \c@secnumdepth >\z@
          \thesection. \ %
        \fi
        ##1}}}}
\else
  \def\ps@headings{%
    \let\@oddfoot\@empty
    \def\@oddhead{{\slshape\rightmark}\hfil\thepage}%
    \let\@mkboth\markboth
    \def\chaptermark##1{%
      \markright {\MakeUppercase{%
        \ifnum \c@secnumdepth >\m@ne
            \@chapapp\ \thechapter. \ %
        \fi
        ##1}}}}
\fi
\def\ps@myheadings{%
    \let\@oddfoot\@empty\let\@evenfoot\@empty
    \def\@evenhead{\thepage\hfil\slshape\leftmark}%
    \def\@oddhead{{\slshape\rightmark}\hfil\thepage}%
    \let\@mkboth\@gobbletwo
    \let\chaptermark\@gobble
    \let\sectionmark\@gobble
    }
%    \end{macrocode}
% Items required for the title page.
%    \begin{macrocode}
\newcommand{\employer}[1]{\ifx\empty#1\else\gdef\@employer{#1}\fi}
\newcommand{\department}[1]{\ifx\empty#1\else\gdef\@department{#1}\fi}
\renewcommand{\author}[1]{\ifx\empty#1\else\gdef\@author{#1}\fi}
\renewcommand{\title}[1]{\ifx\empty#1\else\gdef\@title{#1}\fi}
\newcommand{\gdate}[1]{\ifx\empty#1\else\gdef\@gdate{#1}\fi}

\global\def\@employer{\mbox{}}
\global\def\@degree{\mbox{}}
\global\def\@author{\mbox{}}
\global\def\@title{\mbox{}}
\global\def\@gdate{\number\date} % Default date is current date
%    \end{macrocode}
% Here |\maketitle| is changed to follow the KU guidelines.  It includes its own
% disclaimer page.
% The |\maketitle| command takes one argument to specify what type of title page 
% to generate. 
%    \begin{macrocode}
\newcommand{\maketitle}[1][coop]
{
\begin{titlepage}
   \setcounter{page}{1}
%    \end{macrocode}
% Set the line spacing to 1 for the title page.
%    \begin{macrocode}
   \begin{spacing}{1} 
   \begin{center}
      \mbox{}
      \textbf{\MakeUppercase{\@title}}\\
      \vspace{2em}
      \ifstrequal{#1}{coop}
      {A Co-op Thesis written for \\
       \vspace{1em}
       \textbf{\@employer} \\}
      {}
      \ifstrequal{#1}{research}
      {A Research Thesis written for \\
       \vspace{1em}
       \textbf{KETTERING UNIVERSITY} \\}
      {}
      \ifstrequal{#1}{pp}
      {A Professional Practice Thesis written for \\
       \vspace{1em}
       \textbf{\@employer} \\}
      {}
      \ifstrequal{#1}{eship}
      {An E-Ship Thesis written by \\
       \vspace{1em}
       \textbf{\MakeUppercase{{\@author}}} \\}
      {}
      \vspace{1em}
      and submitted to \\
      \vspace{1em}
      \textbf{KETTERING UNIVERSITY} \\
      \vspace{3em}
      in partial fulfillment \\
      of the requirements for the \\
      degree of \\
      \vspace{1em}
      \textbf{BACHELOR OF SCIENCE IN \MakeUppercase{\@department}} \\
      \vspace{3em}
      \ifstrequal{#1}{eship}
      {}
      {by \\
       \vspace{1em}
       \textbf{\MakeUppercase{{\@author}}}\\
      }
      \vspace{1em}
      {\@gdate} \\
      \vfill
   \end{center}

   \begin{flushright}


   \begin{tabular}{@{}l}
   \\ \hline
   Student    \\
   \\
   \\ \hline
   \ifstrequal{#1}{coop}{Employer Thesis Advisor~~~~~~~~~~~~~~~~~~~~~~~~~~~~~~~~~~~~}{}%
\ifstrequal{#1}{research}{Faculty Thesis Advisor~~~~~~~~~~~~~~~~~~~~~~~~~~~~~~~~~~~~}{}%
\ifstrequal{#1}{pp}{Company Liaison~~~~~~~~~~~~~~~~~~~~~~~~~~~~~~~~~~~~}{}%
\ifstrequal{#1}{eship}{Faculty Thesis Advisor~~~~~~~~~~~~~~~~~~~~~~~~~~~~~~~~~~~~}{} \\
   \\
   \\ \hline   
   \ifstrequal{#1}{coop}{Faculty Thesis Advisor}{}%
\ifstrequal{#1}{research}{Committee Member}{}%
\ifstrequal{#1}{pp}{Faculty Thesis Advisor}{}%
\ifstrequal{#1}{eship}{Committee Member}{} \\
   \end{tabular}


   \end{flushright}
   \end{spacing}
   \end{titlepage}
   \setcounter{page}{2}
   \setcounter{secnumdepth}{-1}

   \chapter{DISCLAIMER}
   \begin{doublespace}
   \par
    This thesis is being submitted as partial fulfillment of the
    graduation requirements of Kettering University needed
    to obtain a Bachelor of Science in {\@department} Degree.
    \par
    The conclusions and opinions expressed in this thesis are those 
    of the student author and do not necessarily represent the position 
    of Kettering University or anyone else affiliated with this 
    culminating undergraduate experience.
   \end{doublespace}
}%\maketitle

\newcommand*\chaptermark[1]{}
%    \end{macrocode}
% Chapters are the only numbered objects, and they use Roman numerals.
% Roman numerals are set below.
%    \begin{macrocode}
\setcounter{secnumdepth}{0}
\setcounter{secnumdepth}{0}
\newcounter {part}
\newcounter {chapter}
\newcounter {section}[chapter]
\newcounter {subsection}[section]
\newcounter {subsubsection}[subsection]
\newcounter {paragraph}[subsubsection]
\newcounter {subparagraph}[paragraph]
\renewcommand \thepart {\@Roman\c@part}
\renewcommand \thechapter {\@Roman\c@chapter}
\renewcommand \thesection {\thechapter.\@arabic\c@section}
\renewcommand\thesubsection   {\thesection.\@arabic\c@subsection}
\renewcommand\thesubsubsection{\thesubsection .\@arabic\c@subsubsection}
\renewcommand\theparagraph    {\thesubsubsection.\@arabic\c@paragraph}
\renewcommand\thesubparagraph {\theparagraph.\@arabic\c@subparagraph}
\newcommand\@chapapp{\chaptername}
\newcommand\part{%
  \if@openright
    \cleardoublepage
  \else
    \clearpage
  \fi
  \thispagestyle{plain}%
  \if@twocolumn
    \onecolumn
    \@tempswatrue
  \else
    \@tempswafalse
  \fi
  \null\vfil
  \secdef\@part\@spart}

\def\@part[#1]#2{%
    \ifnum \c@secnumdepth >-2\relax
      \refstepcounter{part}%
      \addcontentsline{toc}{part}{\thepart\hspace{1em}#1}%
    \else
      \addcontentsline{toc}{part}{#1}%
    \fi
    \markboth{}{}%
    {\centering
     \interlinepenalty \@M
     \normalfont
     \ifnum \c@secnumdepth >-2\relax
       \huge\bfseries \partname~\thepart
       \par
       \vskip 20\p@
     \fi
     \Huge \bfseries #2\par}%
    \@endpart}
\def\@spart#1{%
    {\centering
     \interlinepenalty \@M
     \normalfont
     \Huge \bfseries #1\par}%
    \@endpart}
\def\@endpart{\vfil\newpage
              \if@twoside
                \null
                \thispagestyle{empty}%
                \newpage
              \fi
              \if@tempswa
                \twocolumn
              \fi}
\newcommand\chapter{\if@openright\cleardoublepage\else\clearpage\fi
                    \thispagestyle{plain}%
                    \global\@topnum\z@
                    \@afterindentfalse
                    \secdef\@chapter\@schapter}
%    \end{macrocode}
% We put a period after the chapter number in the table of contents.
% (e.g. I.\hspace{.5in}INTRODUCTION)
%    \begin{macrocode}
\def\@chapter[#1]#2{\ifnum \c@secnumdepth >\m@ne
                         \refstepcounter{chapter}%
                         \typeout{\@chapapp\space\thechapter.}%
                         \addcontentsline{toc}{chapter}%
                                   {\protect\numberline{\thechapter\@.}\MakeUppercase{#1}}%
                    \else
                      \addcontentsline{toc}{chapter}{\MakeUppercase{#1}}%
                    \fi
                    \chaptermark{#1}%
                    \if@twocolumn
                      \@topnewpage[\@makechapterhead{#2}]%
                    \else
                      \@makechapterhead{#2}%
                      \@afterheading
                    \fi}
%    \end{macrocode}
% We do simplistic Chapter Headers.  Note that these fix the table of
% contents as well.
%    \begin{macrocode}
\def\@makechapterhead#1{%
  \vspace*{1in}%
  {\parindent \z@ \centering \normalfont
    \ifnum \c@secnumdepth >\m@ne
        \bfseries \thechapter.\enspace
    \fi
    \bfseries\MakeUppercase{#1}\par\nobreak
    \vskip 40\p@
  }}
\def\@makeschapterhead#1{%
  \vspace*{1in}%
  {\parindent \z@ \centering
    \normalfont
    \interlinepenalty\@M
    \bfseries\MakeUppercase{#1}\par\nobreak
    \vskip 40\p@
  }}
\def\@schapter#1{\if@twocolumn
                   \@topnewpage[\@makeschapterhead{#1}]%
                 \else
                   \@makeschapterhead{#1}%
                   \@afterheading
                 \fi}
%    \end{macrocode}
% The format requires underlined section headers.
%    \begin{macrocode}                      
\newcommand\section{\@startsection{section}{1}{0mm}%
                                    {3.5ex \@plus 1ex \@minus .2ex}%
                                    {2.3ex \@plus.2ex}%
                                    {\raggedright\normalsize\bfseries\underline}}%
\newcommand\subsection{\@startsection{subsection}{2}{0mm}%
                                     {3.25ex\@plus 1ex \@minus .2ex}%
                                     {1.5ex \@plus .2ex}%
                                     {\normalfont\normalsize\bfseries\underline}}%
\newcommand\subsubsection{\@startsection{subsubsection}{3}{0mm}%
                                     {3.25ex\@plus 1ex \@minus .2ex}%
                                     {1.5ex \@plus .2ex}%
                                     {\normalfont\normalsize\bfseries\underline}}%
\newcommand\paragraph{\@startsection{paragraph}{4}{\z@}%
                                    {3.25ex \@plus1ex \@minus.2ex}%
                                    {-1em}%
                                    {\normalfont\normalsize\bfseries}}
\newcommand\subparagraph{\@startsection{subparagraph}{5}{\parindent}%
                                       {3.25ex \@plus1ex \@minus .2ex}%
                                       {-1em}%
                                      {\normalfont\normalsize\bfseries}}
\if@twocolumn
  \setlength\leftmargini  {2em}
\else
  \setlength\leftmargini  {2.5em}
\fi
\leftmargin  \leftmargini
\setlength\leftmarginii  {2.2em}
\setlength\leftmarginiii {1.87em}
\setlength\leftmarginiv  {1.7em}
\if@twocolumn
  \setlength\leftmarginv  {.5em}
  \setlength\leftmarginvi {.5em}
\else
  \setlength\leftmarginv  {1em}
  \setlength\leftmarginvi {1em}
\fi
\setlength  \labelsep  {.5em}
\setlength  \labelwidth{\leftmargini}
\addtolength\labelwidth{-\labelsep}
\@beginparpenalty -\@lowpenalty
\@endparpenalty   -\@lowpenalty
\@itempenalty     -\@lowpenalty
\renewcommand\theenumi{\@arabic\c@enumi}
\renewcommand\theenumii{\@alph\c@enumii}
\renewcommand\theenumiii{\@roman\c@enumiii}
\renewcommand\theenumiv{\@Alph\c@enumiv}
\newcommand\labelenumi{\theenumi.}
\newcommand\labelenumii{(\theenumii)}
\newcommand\labelenumiii{\theenumiii.}
\newcommand\labelenumiv{\theenumiv.}
\renewcommand\p@enumii{\theenumi}
\renewcommand\p@enumiii{\theenumi(\theenumii)}
\renewcommand\p@enumiv{\p@enumiii\theenumiii}
\newcommand\labelitemi{\textbullet}
\newcommand\labelitemii{\normalfont\bfseries \textendash}
\newcommand\labelitemiii{\textasteriskcentered}
\newcommand\labelitemiv{\textperiodcentered}
\newenvironment{description}
               {\singlespacing\begin{list}{}{%
                \renewcommand{\makelabel}[1]{\normalfont ##1:\hfil}%
                \setlength{\labelwidth}{1.5 in}%
                \setlength{\leftmargin}{\labelwidth+\labelsep}}}
               {\end{list}\doublespacing}
\newcommand*\descriptionlabel[1]{\hspace\labelsep
                                 \normalfont #1:\hfil}
\newenvironment{preface}
{
\setcounter{secnumdepth}{-1}
\chapter{\prefacename}\begin{doublespacing}
}
{\end{doublespacing}\setcounter{secnumdepth}{0}}
\newenvironment{verse}
               {\let\\\@centercr
                \list{}{\itemsep      \z@
                        \itemindent   -1.5em%
                        \listparindent\itemindent
                        \rightmargin  \leftmargin
                        \advance\leftmargin 1.5em}%
                \item\relax}
               {\endlist}
\newenvironment{quotation}
               {\list{}{\listparindent 1.5em%
                        \itemindent    \listparindent
                        \rightmargin   \leftmargin
                        \parsep        \z@ \@plus\p@}%
                \item\relax}
               {\endlist}
\newenvironment{quote}
               {\list{}{\rightmargin\leftmargin}%
                \item\relax}
               {\endlist}
\if@compatibility
\newenvironment{titlepage}
    {%
      \if@twocolumn
        \@restonecoltrue\onecolumn
      \else
        \@restonecolfalse\newpage
      \fi
      \thispagestyle{empty}%
      \setcounter{page}\z@
    }%
    {\if@restonecol\twocolumn \else \newpage \fi
    }
\else
\newenvironment{titlepage}
    {%
      \if@twocolumn
        \@restonecoltrue\onecolumn
      \else
        \@restonecolfalse\newpage
      \fi
      \thispagestyle{empty}%
      \setcounter{page}\@ne
    }%
    {\if@restonecol\twocolumn \else \newpage \fi
     \if@twoside\else
        \setcounter{page}\@ne
     \fi
    }
\fi
%    \end{macrocode}
% The appendix command has many tasks.  Note that right now, if you have only
% one appendix, you're on your own.   A lot of this was ripped from
% Peter Wilson's appendix class.  The only reason I ripped it was
% because his system provides no facility for telling it whether
% chapter or section starts an appendix for a non-standard class.
% Since this needs to be fairly easy for fellow students (E.g. no
% editing other packages if I can help it), I rewrote and inserted it here.
%    \begin{macrocode}
\newcommand\appendix{\par
  \thispagestyle{plain}%
  \if@twocolumn\onecolumn\@tempswatrue\else\@tempswafalse\fi
  \markboth{}{}%
  {\setcounter{secnumdepth}{-1}
  \chapter{\appendixtocname}
  \setcounter{secnumdepth}{0}}%
  \vfil\newpage
  \if@twoside
    \if@openright
      \null
      \thispagestyle{empty}%
      \newpage
    \fi
  \fi
  \if@tempswa
    \twocolumn
  \fi

  \setcounter{section}{0}
  \setcounter{chapter}{0}
  \renewcommand\@chapapp{\appendixname}
  \renewcommand\thechapter{\@Alph\c@chapter}
  \let\oldacl@pp=\addcontentsline
  \def\addcontentsline##1##2##3{%
    \ifthenelse{\equal{##1}{toc}}{%
      \ifthenelse{\equal{##2}{chapter}}{%
        \addtocontents{##1}{\protect\contentsline{##2}%
                            {\appendixname\space ##3}{\thepage}}}%
      {\oldacl@pp{##1}{##2}{##3}}}%
    {\oldacl@pp{##1}{##2}{##3}}}
%    \end{macrocode}
% \textbf{Note:} Still in |\appendix|.\\
% Note we have that darned colon in for appendicies in the TOC.
% Also note the ugly kludge being used to remove vertical spacing
% between appendix names.
%    \begin{macrocode}
\def\@chapter[##1]##2{\ifnum \c@secnumdepth >\m@ne
                         \refstepcounter{chapter}%
                         \typeout{\@chapapp\space\thechapter}%
                         \addcontentsline{toc}{section}%
                                   {\@chapapp\space\thechapter: ##1}%
                    \else
                      \addcontentsline{toc}{chapter}{##1}%
                    \fi
                    \chaptermark{##1}%
                      \if@twocolumn
                      \@topnewpage[\@makechapterhead{##2}]%
                    \else
                      \@makechapterhead{##2}%                      
                      \@afterheading
                    \fi}
%    \end{macrocode}
% Now things are redefined a special page just for the chapter
% heading.
%    \begin{macrocode}
\def\@makechapterhead##1{%
  \vspace*{1in}%
  {\parindent \z@ \centering \normalfont
    \ifnum \c@secnumdepth >\m@ne
        \bfseries \@chapapp\  \thechapter\@\par
    \fi
    \vskip 40\p@
    \bfseries \MakeUppercase##1\par\nobreak
    \clearpage
  }}
}
\setlength\arraycolsep{5\p@}
\setlength\tabcolsep{6\p@}
\setlength\arrayrulewidth{.4\p@}
\setlength\doublerulesep{2\p@}
\setlength\tabbingsep{\labelsep}
\skip\@mpfootins = \skip\footins
\setlength\fboxsep{3\p@}
\setlength\fboxrule{.4\p@}
\@addtoreset {equation}{chapter}
\renewcommand\theequation
  {\ifnum \c@chapter>\z@ \thechapter.\fi \@arabic\c@equation}
%    \end{macrocode}
% Here we reconfigure lists of figures and tables to appear in table of
% contents.  Also, parts of report.cls were removed to
% reconfigure figure counter to number for the entire document rather than per
% chapter.
%    \begin{macrocode}
\newcounter{figure}
\def\fps@figure{tbp}
\def\ftype@figure{1}
\def\ext@figure{lof}
\def\fnum@figure{\underline{\figurename~\thefigure.}}
\newenvironment{figure}
               {\@float{figure}}
               {\end@float}
\newenvironment{figure*}
               {\@dblfloat{figure}}
               {\end@dblfloat}
%    \end{macrocode}
% Reconfigured table counter to number for the entire document, not per
% chapter.  (Bits of report.cls removed)
%    \begin{macrocode}               
\newcounter{table}
\def\fps@table{tbp}
\def\ftype@table{2}
\def\ext@table{lot}
\def\fnum@table{\underline{\tablename~\thetable.}}
\newenvironment{table}
               {\@float{table}}
               {\end@float}
\newenvironment{table*}
               {\@dblfloat{table}}
               {\end@dblfloat}
\newlength\abovecaptionskip
\newlength\belowcaptionskip
\setlength\abovecaptionskip{10\p@}
\setlength\belowcaptionskip{0\p@}
\long\def\@makecaption#1#2{%
  \vskip\abovecaptionskip
  \sbox\@tempboxa{#1 #2}%
  \ifdim \wd\@tempboxa >\hsize
    #1 #2\par
  \else
    \global \@minipagefalse
    \hb@xt@\hsize{\hfil\box\@tempboxa\hfil}%
  \fi
  \vskip\belowcaptionskip}
\DeclareOldFontCommand{\rm}{\normalfont\rmfamily}{\mathrm}
\DeclareOldFontCommand{\sf}{\normalfont\sffamily}{\mathsf}
\DeclareOldFontCommand{\tt}{\normalfont\ttfamily}{\mathtt}
\DeclareOldFontCommand{\bf}{\normalfont\bfseries}{\mathbf}
\DeclareOldFontCommand{\it}{\normalfont\itshape}{\mathit}
\DeclareOldFontCommand{\sl}{\normalfont\slshape}{\@nomath\sl}
\DeclareOldFontCommand{\sc}{\normalfont\scshape}{\@nomath\sc}
\DeclareRobustCommand*\cal{\@fontswitch\relax\mathcal}
\DeclareRobustCommand*\mit{\@fontswitch\relax\mathnormal}
\newcommand\@pnumwidth{1.55em}
\newcommand\@tocrmarg{2.55em}
\newcommand\@dotsep{4.5}
\setcounter{tocdepth}{2}
\newcommand\tableofcontents{%
    \if@twocolumn
      \@restonecoltrue\onecolumn
    \else
      \@restonecolfalse
    \fi
    \chapter*{\contentsname
        \@mkboth{%
           \MakeUppercase\contentsname}{\MakeUppercase\contentsname}}%
    \@starttoc{toc}%
    \if@restonecol\twocolumn\fi
    }
\newcommand*\l@part[2]{%
  \ifnum \c@tocdepth >-2\relax
    \addpenalty{-\@highpenalty}%
    \addvspace{2.25em \@plus\p@}%
    \begingroup
      \parindent \z@ \rightskip \@pnumwidth
      \parfillskip -\@pnumwidth
      {\leavevmode
       \large \bfseries #1\hfil \hb@xt@\@pnumwidth{\hss #2}}\par
       \nobreak
         \global\@nobreaktrue
         \everypar{\global\@nobreakfalse\everypar{}}%
    \endgroup
  \fi}
%    \end{macrocode}
% Here we setup table of contents.
%    \begin{macrocode}
\newcommand*\l@chapter{\vspace{1em}\@dottedtocline{0}{0em}{4em}}
\newcommand*\l@section{\@dottedtocline{1}{5.5em}{2.3em}}
\newcommand*\l@subsection{\@dottedtocline{2}{7.8em}{3.2em}}
\newcommand*\l@subsubsection{\@dottedtocline{3}{11.0em}{4.1em}}
\newcommand*\l@paragraph{\@dottedtocline{4}{14em}{5em}}
\newcommand*\l@subparagraph{\@dottedtocline{5}{16em}{6em}}
%    \end{macrocode}
% List of illustrations modified to appear in the table of contents.
%    \begin{macrocode}
\newcommand\listofillustrations{%
    \setcounter{secnumdepth}{-1}    
    \chapter{\listillusname%
      \@mkboth{\MakeUppercase\listillusname}%
              {\MakeUppercase\listillusname}}%
\begin{doublespace}%
\begin{raggedright}%
\textbf{\underline{Figures}\hfill\underline{Page}}%
\end{raggedright}%
\@starttoc{lof}%
\vspace{2em}
\begin{raggedright}%
\textbf{\underline{Tables}\hfill\underline{Page}}%
\end{raggedright}%
\@starttoc{lot}%
\end{doublespace}
\setcounter{secnumdepth}{0}    
}
\newcommand*\l@figure{\@dottedtocline{1}{0em}{2.3em}}
\let\l@table\l@figure
\newdimen\bibindent
\setlength\bibindent{1.5em}
\newenvironment{thebibliography}[1]
     {\chapter*{\bibname
        \@mkboth{\MakeUppercase\bibname}{\MakeUppercase\bibname}}%
      \list{\@biblabel{\@arabic\c@enumiv}}%
           {\settowidth\labelwidth{\@biblabel{#1}}%
            \leftmargin\labelwidth
            \advance\leftmargin\labelsep
            \@openbib@code
            \usecounter{enumiv}%
            \let\p@enumiv\@empty
            \renewcommand\theenumiv{\@arabic\c@enumiv}}%
      \sloppy
      \clubpenalty4000
      \@clubpenalty \clubpenalty
      \widowpenalty4000%
      \sfcode`\.\@m}
     {\def\@noitemerr
       {\@latex@warning{Empty `thebibliography' environment}}%
      \endlist}
\newcommand\newblock{\hskip .11em\@plus.33em\@minus.07em}
\let\@openbib@code\@empty
\newenvironment{theindex}
               {\if@twocolumn
                  \@restonecolfalse
                \else
                  \@restonecoltrue
                \fi
                \columnseprule \z@
                \columnsep 35\p@
                \twocolumn[\@makeschapterhead{\indexname}]%
                \@mkboth{\MakeUppercase\indexname}%
                        {\MakeUppercase\indexname}%
                \thispagestyle{plain}\parindent\z@
                \parskip\z@ \@plus .3\p@\relax
                \let\item\@idxitem}
               {\if@restonecol\onecolumn\else\clearpage\fi}
\newcommand\@idxitem{\par\hangindent 40\p@}
\newcommand\subitem{\@idxitem \hspace*{20\p@}}
\newcommand\subsubitem{\@idxitem \hspace*{30\p@}}
\newcommand\indexspace{\par \vskip 10\p@ \@plus5\p@ \@minus3\p@\relax}
\renewcommand\footnoterule{%
  \kern-3\p@
  \hrule\@width.4\columnwidth
  \kern2.6\p@}
\@addtoreset{footnote}{chapter}
\newcommand\@makefntext[1]{%
    \parindent 1em%
    \noindent
    \hb@xt@1.8em{\hss\@makefnmark}#1}
%    \end{macrocode}
% Kettering requires that these be all caps.  While we do have
% automatic uppercasing in spots, it's best to cover all the bases.
%    \begin{macrocode}
\newcommand\contentsname{TABLE OF CONTENTS}
\newcommand\listillusname{LIST OF ILLUSTRATIONS}
\newcommand\bibname{REFERENCES}             % APA format!
\newcommand\indexname{INDEX}
\newcommand\figurename{Figure}
\newcommand\tablename{Table}
\newcommand\partname{PART}
\newcommand\chaptername{Chapter}
\newcommand\appendixname{APPENDIX}
\newcommand\appendixtocname{APPENDICES}
\newcommand\prefacename{PREFACE}
\def\today{\ifcase\month\or
  January\or February\or March\or April\or May\or June\or
  July\or August\or September\or October\or November\or December\fi
  \space\number\day, \number\year}
\setlength\columnsep{10\p@}
\setlength\columnseprule{0\p@}
\pagestyle{plain}
%    \end{macrocode}
% The thesis begins with Roman page numerals, and continues with Arabic.
% Some sections in the front are double-spaced.
%    \begin{macrocode}
\pagenumbering{roman}
\newcommand\begincontent{
\clearpage
\pagenumbering{arabic}
\doublespacing
}
%    \end{macrocode}
% Margin Setup
%    \begin{macrocode}
\setlength{\oddsidemargin}{.5 in}
\setlength{\footskip}{.5 in}
\setlength{\textheight}{9 in}
\setlength{\textwidth}{6 in}
\setlength{\topmargin}{0pt}
\setlength{\headheight}{0pt}
\setlength{\headsep}{0pt}
\setlength{\voffset}{0pt}
\setlength{\hoffset}{0pt}
\setlength{\textheight}{9in}


\if@twoside
\else
  \raggedbottom
\fi
\if@twocolumn
  \twocolumn
  \sloppy
  \flushbottom
\else
  \onecolumn
\fi
%    \end{macrocode}
%Rather than stomp on the documentation, I'll simply indicate that it's right
%below.  Note that I have altered the file somewhat.  If I was less lazy, I'd pull
%out the unneccesary bits.
%    \begin{macrocode}
%
% Save file as: ENDNOTES.STY
% modified by --bg (B.Gaulle) 09/14/94 for:
%               1) replace � (why a 8bit char here?) by ^ as a default. 
%               2) force \catcode of > to be 12 (implied by \@doanenot).
%          by --bg again 03/22/95 for:
%               3) reseting appropriate catcode of > in case it were
%                  used as an active char before \@endanenote (was
%                  pointed by Ch. Pallier).
%%%% free to distribute by John_Lavagnino@Brown.edu on Thu, 23 Mar 1995 
%%%%%%%%%%%%%%%%%%%%%%%%%%%%%%%%%%%%%%%%%%%%%%%%%%%%%%%%%%%%%%%%%
%
%       ****************************************
%       *              ENDNOTES                *
%       ****************************************
%
%  Date of this version: 24 September 1991.
%
%  Based on the FOOTNOTES section of
%  LATEX.TEX (VERSION 2.09 - RELEASE OF 19 April 1986), with
%  "footnote" changed to "endnote" and "fn" changed to "en" (where
%  appropriate), with all the minipage stuff pulled out, and with
%  some small changes for the different operation of endnotes.
%
%  Uses an extra external file, with .ENT extension, to hold the
%  text of the endnotes.  This may be deleted after the run; a new
%  version is generated each time.
%
%  This code does not obey \nofiles.  Perhaps it should.
%
%   John Lavagnino (lav@brandeis.bitnet), 9/23/88
%   Department of English and American Literature,
%      Brandeis University
%
%  To turn all the footnotes in your documents into endnotes, say
%
%     \let\footnote=\endnote
%
%  in your preamble, and then add something like
%
%     \newpage
%     \begingroup
%     \parindent 0pt
%     \parskip 2ex
%     \def\enotesize{\normalsize}
%     \theendnotes
%     \endgroup
%
% as the last thing in your document.
%
%       ****************************************
%       *            CHANGE LOG                *
%       ****************************************
%
% JL  Modified to include \addtoendnotes.  JL, 10/22/89.
%
% JK  Modification by J"org Knappen 25. 2. 1991:
% JK
% JK  Introduced \notesname in the spirit of international \LaTeX.
% JK  \notesname is set per default to be {Notes}, but can easily
% JK  be redifined, e.g. for german language
% JK  \renewcommand{\notesname}{Anmerkungen}
%
% DW Modification by Dominik Wujastyk, London, 19 September 1991:
% DW
% DW Moved the line
% DW         \edef\@currentlabel{\csname p@endnote\endcsname\@theenmark}
% DW out of the definition of \@endnotetext and into the definition
% DW of \@doanenote so that \label and \ref commands work correctly in
% DW endnotes.  Otherwise, the \label just pointed to the last section
% DW heading (or whatever) preceding the \theendnotes command.
%
% JL Revised documentation and macros.  24 Sept 1991.
%
%       ****************************************
%       *        ENDNOTE COMMANDS              *
%       ****************************************
%
%
%   \endnote{NOTE}       : User command to insert a endnote.
%
%   \endnote[NUM]{NOTE}  : User command to insert a endnote numbered
%                           NUM, where NUM is a number -- 1, 2,
%                           etc.  For example, if endnotes are numbered
%                           *, **, etc. within pages, then  \endnote[2]{...}
%                           produces endnote '**'.  This command does  not
%                           step the endnote counter.
%
%   \endnotemark[NUM]    : Command to produce just the endnote mark in
%                           the text, but no endnote.  With no argument,
%                           it steps the endnote counter before  generating
%                           the mark.
%
%   \endnotetext[NUM]{TEXT} : Command to produce the endnote but no
%                              mark.  \endnote is equivalent to
%                              \endnotemark \endnotetext .
%
%   \addtoendnotes{TEXT} : Command to add text or commands to current
%                              endnotes file: for inserting headings,
%                              pagebreaks, and the like into endnotes
%                              sections.  TEXT a moving argument:
%                              \protect required for fragile commands.
%
%       ****************************************
%       *        ENDNOTE USER COMMANDS         *
%       ****************************************
%
%   Endnotes use the following parameters, similar to those relating
%   to footnotes:
%
%   \enotesize   : Size-changing command for endnotes.
%
%   \theendnote : In usual LaTeX style, produces the endnote number.
%
%   \@theenmark : Holds the current endnote's mark--e.g., \dag or '1' or 'a'.
%
%   \@makeenmark : A macro to generate the endnote marker from  \@theenmark
%                  The default definition is \hbox{$^\@theenmark$}.
%
%   \@makeentext{NOTE} :
%        Must produce the actual endnote, using \@theenmark as the mark
%        of the endnote and NOTE as the text.  It is called when effectively
%        inside a \parbox, with \hsize = \columnwidth.  For example, it might
%        be as simple as
%               $^{\@theenmark}$ NOTE
%
%
%       ****************************************
%       *        ENDNOTE PSEUDOCODE            *
%       ****************************************
%
% \endnote{NOTE}  ==
%  BEGIN
%    \stepcounter{endnote}
%    \@theenmark :=G eval (\theendnote)
%    \@endnotemark
%    \@endnotetext{NOTE}
%  END
%
% \endnote[NUM]{NOTE} ==
%  BEGIN
%    begingroup
%       counter endnote :=L NUM
%       \@theenmark :=G eval (\theendnote)
%    endgroup
%    \@endnotemark
%    \@endnotetext{NOTE}
%  END
%
% \@endnotetext{NOTE} ==
%  BEGIN
%    write to \@enotes file: "\@doanenote{ENDNOTE MARK}"
%    begingroup
%       \next := NOTE
%       set \newlinechar for \write to \space
%       write to \@enotes file: \meaning\next
%               (that is, "macro:->NOTE)
%    endgroup
%  END
%
% \addtoendnotes{TEXT} ==
%  BEGIN
%    open endnotes file if not already open
%    begingroup
%       let \protect to \string
%       set \newlinechar for \write to \space
%       write TEXT to \@enotes file
%    endgroup
%  END
%
% \endnotemark      ==
%  BEGIN \stepcounter{endnote}
%        \@theenmark :=G eval(\theendnote)
%        \@endnotemark
%  END
%
% \endnotemark[NUM] ==
%   BEGIN
%       begingroup
%         endnote counter :=L NUM
%        \@theenmark :=G eval(\theendnote)
%       endgroup
%       \@endnotemark
%   END
%
% \@endnotemark ==
%   BEGIN
%    \leavevmode
%    IF hmode THEN \@x@sf := \the\spacefactor FI
%    \@makeenmark          % put number in main text
%    IF hmode THEN \spacefactor := \@x@sf FI
%   END
%
% \endnotetext      ==
%    BEGIN \@theenmark :=G eval (\theendnote)
%          \@endnotetext
%    END
%
% \endnotetext[NUM] ==
%    BEGIN begingroup  counter endnote :=L NUM
%                      \@theenmark :=G eval (\theendnote)
%          endgroup
%          \@endnotetext
%    END
%
%       ****************************************
%       *           ENDNOTE MACROS             *
%       ****************************************
%

\@definecounter{endnote}
\def\theendnote{\arabic{endnote}}

% Default definition
\def\@makeenmark{\hbox{$^{\@theenmark}$}}

\newdimen\endnotesep

\def\endnote{\@ifnextchar[{\@xendnote}{\stepcounter
   {endnote}\xdef\@theenmark{\theendnote}\@endnotemark\@endnotetext}}

\def\@xendnote[#1]{\begingroup \c@endnote=#1\relax
   \xdef\@theenmark{\theendnote}\endgroup
   \@endnotemark\@endnotetext}

%  Here begins endnote code that's really different from the footnote
% code of LaTeX.

\let\@doanenote=0
\let\@endanenote=0

\newwrite\@enotes
\newif\if@enotesopen \global\@enotesopenfalse

\def\@openenotes{\immediate\openout\@enotes=\jobname.ent\relax
      \global\@enotesopentrue}

%  The stuff with \next and \meaning is a trick from the TeXbook, 382,
% there intended for setting verbatim text, but here used to avoid
% macro expansion when the footnote text is written.  \next will have
% the entire text of the footnote as one long line, which might well
% overflow limits on output line length; the business with \newlinechar
% makes every space become a newline in the \@enotes file, so that all
% of the lines wind up being quite short.

\long\def\@endnotetext#1{%
     \if@enotesopen \else \@openenotes \fi
     \immediate\write\@enotes{\@doanenote{\@theenmark}}%
     \begingroup
        \def\next{#1}%
        \newlinechar='40
        \immediate\write\@enotes{\meaning\next}%
     \endgroup
     \immediate\write\@enotes{\@endanenote}}

% \addtoendnotes works the way the other endnote macros probably should
% have, requiring the use of \protect for fragile commands.

\long\def\addtoendnotes#1{%
     \if@enotesopen \else \@openenotes \fi
     \begingroup
        \newlinechar='40
        \let\protect\string
        \immediate\write\@enotes{#1}%
     \endgroup}

%  End of unique endnote code

\def\endnotemark{\@ifnextchar[{\@xendnotemark
    }{\stepcounter{endnote}\xdef\@theenmark{\theendnote}\@endnotemark}}

\def\@xendnotemark[#1]{\begingroup \c@endnote #1\relax
   \xdef\@theenmark{\theendnote}\endgroup \@endnotemark}

\def\@endnotemark{\leavevmode\ifhmode
  \edef\@x@sf{\the\spacefactor}\fi \@makeenmark
   \ifhmode\spacefactor\@x@sf\fi\relax}

\def\endnotetext{\@ifnextchar
    [{\@xendnotenext}{\xdef\@theenmark{\theendnote}\@endnotetext}}

\def\@xendnotenext[#1]{\begingroup \c@endnote=#1\relax
   \xdef\@theenmark{\theendnote}\endgroup \@endnotetext}


%  \theendnotes actually prints out the endnotes.

% Endnotes auto-renumber at the end of each chapter (or actually, when
% they're printed...)
%  \enoteformat is provided so user can specify some special formatting
% for the endnotes.  It needs to set up the paragraph parameters, start
% the paragraph, and print the label.  The \leavemode stuff is to make
% and undo a dummy paragraph, to get around the games \section*
% plays with paragraph indenting.

\def\notesname{NOTES}% <------ AE
\def\enoteheading{\chapter*{\notesname
  \@mkboth{\uppercase{\notesname}}{\uppercase{\notesname}}}%
     \leavevmode\par\vskip-\baselineskip}

\def\enoteformat{\vspace{\baselineskip}\rightskip\z@ \leftskip\z@ \parindent=1.8em
     \leavevmode\llap{\hbox{$^{\@theenmark}$}}}

\def\enotesize{\normalsize}

% The definition of \ETC. is needed only for versions of TeX prior
% to 2.992.  Those versions limited \meaning expansions to 1000
% characters; in 2.992 and beyond there is no limit.  At Brandeis the
% BIGLATEX program changed the code in the token_show procedure of
% TeX to eliminate this problem, but most ``big'' versions of TeX
% will not solve this problem.

\def\theendnotes{\immediate\closeout\@enotes \global\@enotesopenfalse
  \begingroup
    \makeatletter
% the following is to save catcode of ``>'' and restore it in \@endanenote
\edef\@tempa{`\string >}%
\ifnum\catcode\@tempa=11\let\@ResetGT\relax% accepts also that > were active
  \else\edef\@ResetGT{\noexpand\catcode\@tempa=\the\catcode\@tempa}%
\fi%
    \catcode`>=12% char > will be read as char so force it to \catcode 12 --bg\edef\GOfrench{`\string @}% temp def futher correctly defined
    \def\@doanenote##1##2>{\def\@theenmark{##1}\par\begingroup
        \@ResetGT%\catcode`>=13
        \edef\@currentlabel{\csname p@endnote\endcsname\@theenmark} %DW
        \enoteformat}
    \def\@endanenote{\par\endgroup}%
    \def\ETC.{\errmessage{Some long endnotes will be truncated; %
                            use BIGLATEX to avoid this}%
          \def\ETC.{\relax}}
    \clearpage
    \enoteheading
    \enotesize
    \begin{singlespacing}
    \input{\jobname.ent}%
    \end{singlespacing}
  \endgroup
  \setcounter{endnote}{0}}
\endinput
%    \end{macrocode}
% \iffalse
%</main>
% \fi

